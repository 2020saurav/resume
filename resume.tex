\documentclass[margin,10pt]{resume}
\usepackage{hyperref}
\usepackage{graphicx}
\usepackage{setspace}
\usepackage{tabularx}
\linespread{1.3}
\renewcommand{\baselinestretch}{1.2}
\hypersetup{
    colorlinks=false,
    pdfborder={0 0 0},
}
\begin{document}
\name{\Large Saurav Kumar\vspace{3mm}}
\address{\small Final Year Undergraduate\\\small Computer Science and Engineering \\\small Indian Institute of Technology Kanpur}
\address{\small phone: +918765162302\\\small email: ksaurav@iitk.ac.in, 2020saurav@gmail.com\\\small homepage: \url{http://home.iitk.ac.in/~ksaurav}}

\begin{resume}
\section{\mysidestyle Education}
\vspace{2mm}
\begin{table}[h]
\renewcommand{\arraystretch}{1.5}
\vspace{5mm}
\begin{tabularx}{\textwidth}{X X l c c}
\hline

EXAMINATION      & UNIVERSITY & INSTITUTE          & YEAR           & CPI/\%    \\ \hline

Graduation       & IIT Kanpur & IIT Kanpur         & 2016(expected) & 8.1*/10.0 \\ \hline

Intermediate(+2) & CBSE       & Chinmaya Vidyalaya & 2012           & 94.8\%    \\ \hline

Matriculation    & CBSE       & Chinmaya Vidyalaya & 2010           & 10.0/10.0    \\ \hline

\end{tabularx}

{\vspace{2mm}\small * at the end of seventh semester.}
\end{table}

\vspace{-2mm}
\section{\mysidestyle Scholastic\\Achievements}

\begin{list2}
\begin{spacing}{1.4}
\renewcommand{\baselinestretch}{1.3}

\item Secured \textbf{All India Rank 187} in IITJEE-2012 among 4.5 lakh students.
\item Secured \textbf{All India Rank 97} in AIEEE-2012 among 1.2 million students.
\item Qualified for \textbf{ACM ICPC} (International Collegiate Programming Contest) \mbox{Kharagpur} Regionals 2014.
\item Qualified for \textbf{ACM ICPC} \mbox{Amritapuri} Regionals 2013.
\item Shortlisted for the \textbf{Aditya Birla Scholarships} in 2012.
\item Awarded \textbf{Best Student (Academics)} Award 2012 by Chinmaya Vidyalaya, Bokaro.
\item Stood National \textbf{top 1\%} in \textbf{CBSE} Senior School Certificate Examination 2012.
\item Declared successful in \textbf{Indian National Mathematical Olympiad} (INMO) in 2011 and in 2012, conducted by Homi Bhabha Centre for Science Education (\textbf{HBCSE}).
\item Secured National \textbf{top 1\%} in National Standard Examinations in
Physics (\textbf {NSEP}) \mbox{2011-12,} \\Chemistry (\textbf {NSEC}) 2011-12 and Astronomy (\textbf{NSEA}) 2011-12 conducted by \textbf{HBCSE}.
\item Secured \textbf{All India Rank 2} in 2009 and \textbf{All India Rank 1} in 2010 in National Cyber Olympiad (\textbf{NCO}) organised by Science Olympiad \mbox{Foundation.}
\item Secured \textbf{All India Rank 10} in 2011 and \textbf{All India Rank 28} in 2012 in National Science Olympiad (\textbf{NSO}) organised by Science Olympiad Foundation.
\item Awarded \textbf{DST Medal} in IGNOU-UNESCO Science Olympiad 2010.
\item Awarded \textbf{KVPY (Kishore Vaigyanik Protsahan Yojna) Fellowship} in 2010
by Department of Science and Technology, Government of India.
\item Awarded \textbf{NTSE Scholarship} based on National Talent Search Examination in 2008 by NCERT,\\ Government of India.
\end{spacing}
\end{list2}
\vspace{-7mm}
\section{\mysidestyle Internships}
\begin{list2}
\item \textbf{LinkedIn, Bangalore}\\
\textsl{Content Filtering Tools} \hfill \emph{May 2015 - July 2015}

	\begin{list3}
	\item Developed two internal tools for Content Filtering Team using Play Framework for Java, RestLi, ParSeq, Couchbase and internal email tool.
	\item The first tool ran all the text spam classifiers and presented the summary with scores and confidences
	\item The second tool identified specific parts of the content which contributed more towards the spam score of the text content, and highlighted them proportional to the score.
	\end{list3}

	\vspace{13mm}

\item \textbf{Aurus Networks, Bangalore}\\
\textsl{SuperProfs Web Application and API Development} \hfill \emph{May 2014 - June 2014}

	\begin{list3}
	\item Development of website and related APIs for SuperProfs.com, an online platform for exam preparation.\vspace{1mm}
	\item Creation of portal for professors to create courses, update profile and view transactions; course review and rating; related APIs for data of users and courses.
	\end{list3}
\end{list2}

\newpage
\section{\mysidestyle Projects and Developments}
\begin{list2}

\item \textbf{[Software Architecture] TBD }\\
	\textsl{Course Project under Dr. T V Prabhakar} \hfill \emph{March 2016}\\
	Lorem ipsum ...

	\vspace{4mm}

\item \textbf{[Cyber Security] Zook .. }\\
	\textsl{Course Project under Dr. Sandeep Shukla} \hfill \emph{Jan 2016 - April 2016}\\
	Lorem ipsum ...

	\vspace{4mm}

\item \textbf{[Computer Vision] TBD}\\
	\textsl{Course Project under Dr. Vinay Namboodiri} \hfill \emph{March 2016}\\
	Lorem ipsum ...

	\vspace{4mm}

\item \textbf{[Distributed Graph Algorithms] Architecture and algorithm for some graph problems}\\
	\textsl{Undergraduate Project under Dr. Arnab Bhattacharya} \hfill \emph{Aug 2015 - Nov 2015}\\
	Lorem ipsum ...

	\vspace{4mm}

\item \textbf{[Functional Programming] AriaDB: A key-value datastore in Haskell}\\
	\textsl{Course Project under Dr. Piyush P Kurur} \hfill \emph{October 2015}\\
	Lorem ipsum ...

	\vspace{4mm}

\item \textbf{[Machine Learning] Sentiment Analysis in Movie Reviews}\\
	\textsl{Course Project under Dr. Harish Karnick} \hfill \emph{April 2015}\\
	The task, from Kaggle, was to predict the sentiment of a movie review from the test set based on the training set that was provided from the IMDb set of movie reviews.

	\vspace{4mm}

\item \textbf{[Distributed B+ Tree] Efficient key distribution technique in distributed B+ Trees}\\
	\textsl{Course Project in Advanced Databases under Dr. Arnab Bhattacharya} \hfill \emph{March 2015}\\
	The aim was to implement a B+ Tree index structure over a distributed multi-node network; and to devise an efficient distribution of nodes and analyze its performance. Efficient key distribution is done using the domain knowledge of queries and mapping more frequent keys to \textit{better} servers.

	\vspace{4mm}

\item \textbf{[Compilers] Compiler for Python}\\
	\textsl{Course Project under Dr. Subhajit Roy} \hfill \emph{Jan 2015 - Apr 2015}\\
	An end-to-end compiler for Python in the MIPS architecture with support for loops, recursions, nested functions and type inference implemented in Python.\\
	 Github Link: \url{http://github.com/2020saurav/py-codegen}

	\vspace{4mm}

\item \textbf{[Operating Systems] NachOS}\\
	\textsl{Course Project under Dr. Mainak Chaudhari} \hfill \emph{Aug 2014 - Nov 2014}\\
	Extended the standard system call library of NachOS; implemented process scheduling algorithms such as UNIX scheduling, Round Robin, Shortest Job First and Non Pre-emptive; implemented page replacement algorithms such as Random Page Allocation, FIFO, LRU and LRU Clock.

	\vspace{4mm}

% \item \textbf{[Hardware Description Language] Implementation of Hilbert Transformation}\\
% 	\textsl{Electronic Circuit Design Competition in Techkriti'14} \hfill \emph{March 2014}\\
% 	Worked actively in a team of 4 to implement Hilbert Transform in Verilog HDL. Used Open Source tool \emph{iverilog} to simulate the hardware, using \emph{Cooley Tookey Algorithm}.\\
% 	Github Link: \url{http://github.com/2020saurav/hilbert}
% 	\vspace{2mm}

% \item \textbf{[Web Development] Website for Crypto'13 - an online game} \\
%     \textsl{Link: \url{http://crypto13.comoj.com}} \hfill \emph{September 2013}\\
% Created the web application and successfully conducted Crypto'13, a competitive event in Intra College Technical Festival \emph{Takneek}. This involved works in HTML, PHP, JS, CSS and MySQL.\vspace{2mm}

% \item \textbf{[Utility Application] ImageBucket}  \\
% \textsl{App development in Yahoo! HackU} \hfill \emph{August 2013}\\
% The app queries local disk based on queries such as number of persons in an image, prominent color of the image, size and date. It uses OpenCV libraries in Python for face-detection and generates a database for each directory. The PHP based query end fetches fast and good results from the database.\\
% Github Link: \url{http://github.com/2020saurav/hacku13}\vspace{2mm}

\item\textbf{[Computer Vision and Robotics] Bot Automation using Raspberry Pi} \\
    \textsl{Summer Project under Programming Club} \hfill \emph{May 2013}\\
	Automated a bot with on-board web-camera to search a green ball and to reach near it using Raspberry Pi's GPIO, OpenCV libraries for image processing in Python.\\
	Github Link: \url{http://github.com/2020saurav/raspi}

\end{list2}

\newpage

\section{\mysidestyle Skills and\\Interests}
\begin{list2}
	\item \emph{\textbf {Languages:}} C++, Python, Java, Haskell, R, PHP, Verilog, Assembly
	\vspace{1mm}
	\item \emph{\textbf{Technologies:}} Play, NodeJS, Yii, SQL, NoSQL
	\vspace{1mm}
	\item \emph{\textbf{Tools:}} Android SDK, \LaTeX, GNU Octave, Git, SVN, OpenCV
\end{list2}


\section{\mysidestyle Key Courses\\undertaken }
\begin{list2}
	\item \emph{\textbf{Core:}} Fundamentals of Computer Science, Data Structure and Algorithms, Logic for Computer Science, Computer Organization, Discrete Maths, Abstract Algebra, Theory of Computation, Operating Systems, Compiler Design, Principles of Databases, Searching and Indexing in Databases, Machine Learning, Functional Programming, Principles of Programming Languages, Game Theory, Software Architecture*, Computer Vision*, Cyber Security*

	\vspace{2mm}

	\item \emph{\textbf{Breadth:}} Multivariable Calculus, Linear Algebra, Ordinary Differential Equation, Complex Analysis, Introduction to Electronics, Probability and Statistics, Introduction to Electrical Engineering
\end{list2}
	{\vspace{-3mm}\hfill\small * to be completed in April 2016.}


\section{\mysidestyle Positions of\\Responsibility }
\vspace{5mm}
\begin{list2}
\item \textbf{Co-ordinator,} \emph{Association of Computing Activities(ACA), IIT Kanpur} \hfill \emph{Aug'14 - July'15}
Organizing League of Programmers lecture series, maintaining Online Judge, organizing events such as \emph{Microsoft code.fun.do}, Happy Hours and departmental freshers.
\vspace{2mm}
\item \textbf{Web-Executive,} \emph{Science \& Technology Council, IIT Kanpur} \hfill \emph{Apr'13 - Mar'14}\\
Development of SnT Blog, and updation of council homepage.

% \item \textbf{Webmaster,} \emph{Counselling Service, IIT Kanpur} \hfill \emph{May'13 - Apr'14}\\
% Maintenance and Updation of Counselling Service Website.
\vspace{2mm}
\item \textbf{Secretary,} \emph{Programming Club, IIT Kanpur}  \hfill \emph{May'13 - Apr'14} \\
Worked with a team of 17 for smooth conduction of various activities of the club and organizing workshops for freshmen for various events.

\end{list2}

\end{resume}
\end{document}
